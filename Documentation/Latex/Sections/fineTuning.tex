
\section{Fine Tuning} 
\begin{flushleft}

When dealing with Images and especially when solving classification problems, most of the time Convolutional Neural Network is a good choice to start with or even the only one that can reach a reasonable amount of performances. However building from scratch a new CNN model is something that takes very hard effort both in terms of time and also in terms of computational requirements. \\
Therefore, a solution to the problem explained above, is to use what it is called "Transfer Learning", that means, instead developing a new model, and the train it, we take an already trained model that works well on some training set for a certain task and then, we adapt it to solve our problem. \\
Even though with Transfer Learning, the issue of creating a new model is no more present, there is still the difficulty of re-training the entire Neural Network on a new dataset to tune the network chosen in order to solve the problem. So, what nowadays is common to do is to :
\begin{itemize}
    \item Select a pre-trained model in the literature that already was trained using a big dataset, and that seems to solve a problem similar to the one that need to be addressed.
    \item Remove the last part of it and replace it with new fully connected layers.
    \item Do not allow the not fully connected layers to be trainable.
    \item Train the Network with the new dataset.
\end{itemize}

Such process is what it is also known as Fine Tuning and it is the technique that was used in this project. 
\end{flushleft}